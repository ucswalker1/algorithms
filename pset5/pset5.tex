\documentclass{article}
\usepackage{amsmath}
\usepackage{amsfonts}
\usepackage{tikz}

\usepackage[shortlabels]{enumitem}

\usepackage{algorithm}
\usepackage{booktabs}
\usepackage{algpseudocode}

\textwidth=7.6in
\textheight=9.9in
\topmargin=-.9in
\headheight=0in
\headsep=.5in
\hoffset=-1.5in
\setlength\parindent{0pt}


\begin{document}

\begin{center}
    \Large{\textbf{Problem Set 5 Solutions}} \\[0.25ex]
    Calvin Walker
\end{center}
\textbf{Problem 1:} 
\begin{enumerate}[a)]
    \item \textcolor{white}{{x}}
    \begin{table}[htbp]
    \centering
    \begin{tabular}{|c|c|c|c|c|c|c|c|c|c|c|}  
    \hline
    & \textbf{s} & \textbf{a} & \textbf{b} & \textbf{c} & \textbf{d} & \textbf{e} & \textbf{f} & \textbf{g} & \textbf{h} & \textbf{t}\\
    \hline
    \textbf{k=1} &0 &4 &5 &8 & $\infty$ & $\infty$ & $\infty$ & $\infty$ & $\infty$ & $\infty$\\
    \hline
    \textbf{k=2} &0 &3 &5 &8 &1 &9 &$\infty$ &$\infty$ &13 &$\infty$ \\
    \hline
    \textbf{k=3} &0 &3 &5 &8 &0 &9 &10 &17 &3 &20\\
    \hline
    \textbf{k=4} &0 &3 &5 &6 &0 &9 &10 &7 &2 &10\\
    \hline
    \textbf{k=5} &0 &3 &5 &6 &0 &7 &10 &6 &2 &9 \\
    \hline
    \textbf{k=6} &0 &3 &5 &6 &0 &7 &8 &6 &2 &9 \\
    \hline
    \textbf{k=7} &0 &3 &5 &4 &0 &7 &8 &6 &2 &9\\
    \hline
    \textbf{k=8} &0 &3 &5 &4 &0 &5 &8 &6 &2 &9 \\
    \hline
    \textbf{k=9} &0 &3 &5 &4 &0 &5 &6 &6 &2 &9 \\
    \hline
    \textbf{k=10} &0 &3 &5 &2 &0 &5 &6 &6 &2 &9 \\
    \hline
    \end{tabular}
    \caption{Simulated Execution of the Bellman-Ford Algorithm}
    \end{table}
    \item For all verticies $v \in V(G) \setminus \{c, e, f\}$, the shortest distance from $s$ to $v$ is properly computed. This is because $c, e$ and $f$ are part of a negative cycle in $G$. Looking at the 10th iteration, we see that we can reach $c$ with a cost of $2$, and $e$ with a cost of $5$. However, $e$ is only a cost of $1$ away from $c$, and both are within $2$ edges of $s$, a clear warning sign of the failure. 
\end{enumerate}
\textbf{Problem 2:} \\[0.75ex]
At each timeslot, we can either stay at the current stage, or travel to another. Let $h'(i, j)$ be the max happiness up to time $j$ at stage $i$. Then we have the following recurrance relation: 
\begin{equation*}
    h'(i, j) = h_{ij} + \max \{h'(i, j - 1),\ \max_{i'}\{h'(i',\ j - 2)\}\}
\end{equation*}
Algorithm: Initialize a $k \times n$ matrix $H'$ such that $h'_{i1} = h_{i1}\ \forall i \in [k]$, and let the other entires of $H'$ be zero. Then, for each time $j \in [2 \dots n]$, iterate over each $i \in [k]$, and let $H'(i, k) = h_{ij} + \max \{H'(i, j - 1),\ \max_{i'}\{H'(i',\ j - 2)\}\}$. Since at times $j \leq 2$ there is no valid entry for $H'(i',\ j - 2)$,
we return $0$ if a call is made outside the bounds of $H'$. Finally, we return the maximum entry in the last column of $H'$, $\max_{i'}H'(i', n)$, to obtain the maximum total enjoyment from attending the music festival. \\[0.5ex]
Runtime: For each time $1$ through $n$, we compute the maximum enjoyment at the $k$ possible stages, each of which must check the $k$ stages at time $j - 2$. So the total runtime is $O(nk^2)$. \\[0.5ex]
Proof of Correctness: For each stage, we can partition the festival schedules that finish by time $j$ at stage $i$ based on whether they were traveling in time $j - 1$ or not. \begin{enumerate}
    \item If they were not traveling at time $j - 1$ then they must be at the stage $i$ at time $j - 1$. These schedules consist of all schedules finishing at stage $i$ at time $j - 1$. The best such schedule has enjoyment $h'(i, j - 1)$.
    \item If they were traveling at time $j - 1$ they can be at any stage $i' \in [k]$ at time $j - 2$. These schedules consist of all schedules finishing at time $j - 2$. The best such schedule has enjoyment $\max_{i'}h'(i', j - 2)$.
\end{enumerate}
Thus, the maximum enjoyment at stage $i$ at time $j$ is equal to $h_{ij}  + \max \{h'(i, j - 1),\ \max_{i'}\{h'(i',\ j - 2)\}\}$
\\[1.0ex]
\textbf{Problem 3:}
\begin{enumerate}[a)]
    \item This is tree, since it is connected and acyclic.
    \item Algorithm: Let $S = (V, E)$ be the directed graph of company $S$. Initialize a set $N = \{CEO\}$ of the employees notified thus far. 
    Then, label each employee vertex with the number of employees they are incharge of (the number of verticies reachable from their vertex). While $N \neq V$, for each employee in $N$, alert the not yet notified employee that they directly supervise with the most employees below them in $S$. Add the alerted employee to $N$.\\[0.5ex]
    Runtime: We can label the employee verticies with the number of employees they are in charge of in linear time using a traversal such as DFS. Then, we can traverse the graph at most $n - 1$ more times (since the CEO is already notified), and for each vertex in $N$, 
    add its child not in $N$ with the most children (employees). So the runtime is $O(n^2)$. \\[0.5ex]
    Explanation: If we have a choice between notifying two employees, and one has more employees than the other, we want to give that employee more days to notify their employees, since notifying everyone in the company is bottle-necked by the part of the company tree with the most verticies. It follows that each employee already notified, should notify their employee with the most employees below them. 
\end{enumerate}
\textbf{Problem 4}: \\[1.0ex]
Algorithm: We can simplify this problem by considering the following subproblem. On the interval $[l, r]$, which pokemon in $p_l \dots p_r$ could win the battle? One way of solving this subproblem would be to consider all the pokemon who can win on $[l + 1, r]$, and seeing if $p_l$ can beat one of these pokemon, or if some of these pokemon can beat $p_l$. Let $W(l, r)$ be the set of possible winners over $[l, r]$. We say $p_i > p_j$ if $p_i$ beats $p_j$. Then, we have the following recurrance relation: 
\begin{equation*}
    W(l, r) = \{p_i \in W(l + 1, r)\ |\ p_i > p_l\} \cup \{p_j \in W(l, r - 1)\ |\ p_j > p_r\}
\end{equation*}
Where $p_r$ and $p_l$ are also added to $W(l, r)$ if they can beat a pokemon in $W(l, r - 1)$ or $W(l + 1, r)$ respectivley. We simply return the result of $W(1, n)$ to obtain a list of all the possible winners of the battle. \\[0.5ex]
Explanation: One way to think about the problem is to consider the circumstances in which pokemon $p_i$ is a possible winner. There must be a possible winner within the intervals to the left and right of him on the line who he can beat. Since we start the recursion from each end of the line, the algorithm effectivley checks both directions for each pokemon. For instance, $p_i$ is only added to the potential winners on $W(i, n)$ if he can beat a pokemon from $W(i + 1, n)$, and only added to $W(1, i)$ if he can beat a pokemon from $W(1, i - 1)$. Similarly, the possible winners from these smaller intervals are only kept if they can beat him. So by the last recursive call, only potential winners on the interval $[1, n]$ are remaining.  \\[0.5ex]
Runtime: Since the result of a single battle can be found in constant time, for some pokemon $p_i$, we can compare them to the pokemon in $W(i + 1, n)$ and $W(1, i - 1)$ in linear time. We do this for each of the $n$ pokemon, so the runtime is $O(n^2)$.

    

\end{document}