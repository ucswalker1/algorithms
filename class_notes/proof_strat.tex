\documentclass{article}
\usepackage{amsmath}
\usepackage{amsfonts}
\usepackage{tikz}
\usepackage{algorithm}
\usepackage{algpseudocode}

\textwidth=7.6in
\textheight=9.9in
\topmargin=-.9in
\headheight=0in
\headsep=.5in
\hoffset=-1.5in
\setlength\parindent{0pt}

\begin{document}

\begin{center}
    \textbf{Theory of Algorithms Proof Strategies} \\[0.25ex]
    Calvin Walker
\end{center}
\textbf{Greedy Algorithms}:  \\[1.0ex]
\underline{Greedy Algorithm Stays Ahead}: At each step, the greedy algorithim does at least as well as any other solution. 
\begin{itemize}
    \item Define $f_j(S)$ to be the performance of schedule $S$ at time-step $j$
    \item Use induction to show that for all $j$, $f_j(S_{greedy}) \geq f_j(S')$ \begin{itemize}
        \item For the inductive step, demonstrate that all choices available to $S'$ are available to $S_{greedy}$, and that $S_{greedy}$ always takes the optimum of these choices. 
    \end{itemize}
    \item We can also use proof by contradiction, by assuming that $j$ is the first index such that $f_j(S_{greedy}) < f_j(S')$ \begin{itemize}
        \item Show directly for base case. 
        \item Use the fact that $f_{j - 1}(S_{greedy}) \geq f_{j - 1}(S')$ to arrive at a contradiction
    \end{itemize}
\end{itemize}
\underline{Exchange Argument}: For any optimal solution $S$, we can iteratively adjust $S$ without affecting performance until it is $S_{greedy}$, so $S_{greedy}$ is also an optimal solution. 
\begin{itemize}
    \item For $i_{j + 1} < i_j$, consider swapping $i_j$ and $i_{j + 1}$
    \item Show that the swap either does not affect some metric of interest, or improves it.
\end{itemize}
\textbf{Divide and Conquer}:


\end{document}